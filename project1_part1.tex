\documentclass[a4paper, 11pt]{article}
\usepackage{comment} % enables the use of multi-line comments (\ifx \fi) 
\usepackage{lipsum} %This package just generates Lorem Ipsum filler text. 
\usepackage{fullpage} % changes the margin

\begin{document}
%Header-Make sure you update this information!!!!
\noindent
\large\textbf{Project1 Part1 Report} \hfill 
\normalsize CS4740 \hfill Team members: Ransen Niu, Weicheng Yu \\

\section*{Problem Statement}
Part 1 of this project uses unsmoothed unigram and bigram models to generate random sentences.

\section*{Preprocessing}
We considered the email addresses at the beginning of the text be irrelevant text and removed them. And we removed the ``bad" punctuations, which are usually irrelevant in generating sentences, including . Then we tokenized the text into an ordered list of words.\newline
\newline
We consider all words are capital-sensitive. Also, we have a set of  ``stop" punctuations which indicate the end of the sentence.

\section*{Unsmoothed Unigram and Sentence Generation}
We build the unigram model by taking counts of each word's frequency divided by the total number of the word tokens. To generate a sentence, we randomly sample words according to the unigram models. For example, a word w that has a unigram model {P(w) = 0.1} would have 10 percent chance to be picked. When one of the ``stop" punctuations appear, the sentence ends.

\subsection*{Sentence Examples}
\subsection*{Sentence Analysis}

\section*{Unsmoothed Bigram and Sentence Generation}
For bigram model, we prepend a beginning marker ``\\begin" and append a stop marker ``\\stop" to the ordered list of words. Then we go through the element (word) in the list two by two to count the frequency of each pair of them. To generate a sentence, we first need to determine the start of the sentence. We choose the beginning word according to the unigram model and the capitalization of the word, i.e. we only consider words that have the first letter 

\subsection*{Sentence Examples}
\subsection*{Sentence Analysis}

\section*{Comparisons between Unigram and Bigram models}
\lipsum[5]

\section*{Analysis \& Testing}
\lipsum[6]

\section*{Final Evaluation}
\lipsum[7]

\section*{Attachments}
%Make sure to change these
Lab Notes, HelloWorld.ic, FooBar.ic
%\fi %comment me out

\begin{thebibliography}{9}
\bibitem{Robotics} Fred G. Martin \emph{Robotics Explorations: A Hands-On Introduction to Engineering}. New Jersey: Prentice Hall.
\bibitem{Flueck}  Flueck, Alexander J. 2005. \emph{ECE 100}[online]. Chicago: Illinois Institute of Technology, Electrical and Computer Engineering Department, 2005 [cited 30
August 2005]. Available from World Wide Web: (http://www.ece.iit.edu/~flueck/ece100).
\end{thebibliography}

\end{document}
